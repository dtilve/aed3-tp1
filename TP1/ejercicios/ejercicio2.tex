\documentclass[10pt,a4paper]{article}
\usepackage[utf8]{inputenc} % para poder usar tildes en archivos UTF-8
\usepackage[spanish]{babel} % para que comandos como \today den el resultado en castellano
\usepackage{a4wide} % márgenes un poco más anchos que lo usual
\usepackage{listings}
\usepackage{color}

\begin{document}

\definecolor{dkgreen}{rgb}{0,0.6,0}
\definecolor{gray}{rgb}{0.5,0.5,0.5}
\definecolor{mauve}{rgb}{0.58,0,0.82}
\lstset{frame=tb,
  language=Java,
  aboveskip=3mm,
  belowskip=3mm,
  showstringspaces=false,
  columns=flexible,
  basicstyle={\small\ttfamily},
  numbers=none,
  numberstyle=\tiny\color{gray},
  keywordstyle=\color{blue},
  commentstyle=\color{dkgreen},
  stringstyle=\color{mauve},
  breaklines=true,
  breakatwhitespace=true,
  tabsize=3
}

\section{Genkidama}

\subsection{Descripcion del problema}
\begin{verbatim}
\end{verbatim}

\subsection{Pseudocódigo}
\begin{verbatim}
\end{verbatim}

\subsection{Cota de complejidad}
\begin{verbatim}
\end{verbatim}

\subsection{Extracto importante de código}
\begin{lstlisting}
\end{lstlisting}

\subsection{Experimentacion}
\begin{verbatim}
\end{verbatim}

\end{document}
