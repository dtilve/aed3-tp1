\documentclass[10pt,a4paper]{article}
\usepackage[utf8]{inputenc} % para poder usar tildes en archivos UTF-8
\usepackage[spanish]{babel} % para que comandos como \today den el resultado en castellano
\usepackage{a4wide} % márgenes un poco más anchos que lo usual
\usepackage{listings}
\usepackage{color}

\begin{document}

\definecolor{dkgreen}{rgb}{0,0.6,0}
\definecolor{gray}{rgb}{0.5,0.5,0.5}
\definecolor{mauve}{rgb}{0.58,0,0.82}
\lstset{frame=tb,
  language=Java,
  aboveskip=3mm,
  belowskip=3mm,
  showstringspaces=false,
  columns=flexible,
  basicstyle={\small\ttfamily},
  numbers=none,
  numberstyle=\tiny\color{gray},
  keywordstyle=\color{blue},
  commentstyle=\color{dkgreen},
  stringstyle=\color{mauve},
  breaklines=true,
  breakatwhitespace=true,
  tabsize=3
}

\section{Kaio Ken}

\subsection{Descripcion del problema}

\begin{verbatim}
	El problema se trata de analizar la cantidad de peleas que debe haber entre un grupo de guerreros, si en cada pelea hay dos bandos a los que pueden pertenecer, para que todos pelen con todos y en cada pelea esten en uno de los dos bandos. Matemáticamente lo que proponemos para resolver en problema es: un conjunto, que se divide en dos una cantidad finita de veces tal que todos los elementos hallan estado en el conjunto opuesto a todos los demás elementos en elgun momento.
	Observamos que la cantidad de peleas es logarítmica respecto a la cantidad de guerreros o elementos del conjunto principal: si n es potencia de 2 en log_2(n) y si n no es potencia de 2 en parte entera(log_2(n))+1. Esto es porque en la primer pelea, si establecemos que n/2 elementos pertenecen al bando 1 y n/2 elementos pertenecen al bando 2, ya tenemos que cada elemento peleo con la mitad del conjunto. Podemos observar que resta que cada elemento pelee contra sus compañeros de equipo actual. Por lo que, repetimos el paso anterior para cada conjunto de n/2 elementos. Y así vamos a obtener en log_2(n) peleas, que cada elemento pelea con todos los demás, porque el conjunto de n elementos se “divide” en dos cada vez; decimos “divide” porque los equipos son dos siempre: bando 1 y bando dos, pero uso la expresión para explicar cómo obtenemos el resultado que todos peleen con todos. 
	Debemos notar que si la cantidad de elementos es potencia de 2, la cantidad de peleas es exactamente el logaritmo en base 2 de n. Pero si n no es potencia de 2, voy a necesitar una pelea más para que todos los elementos completen las peleas, es decir que todos terminen de pelear. Esto sucede porque la cantidad de elementos se encuentra entre 2 potencias de 2 (el. Entre 16=2^n y 32=2^(n+1)) cualesquiera, por lo tanto, las peleas anteriores dividieron a los grupos n veces, quedan menos de la mitad de elementos pelear con la anterior mitad. 
\end{verbatim}

\subsection{Pseudocódigo}
\begin{verbatim}
Datos de entrada: n.
Generamos una matriz M de n * (log(n)+1) posiciones.
En un ciclo que itera i de 0 a log(n) inclusive:
	En un cliclo que itera j de 0 a n-1 inclusive:
		Para la iteracion i asignamos M(i,j) grupo 1 y dos alternadamente a las posiciones, cambiando de grupo cada 2^i casillas.
Imprimimos i. Luego, imprimimos la matriz.
\end{verbatim}

\subsection{Cota de complejidad}
\begin{verbatim}
Cota de complejidad para el algoritmo que proponemos:
Nota: parte entera superior(log(n))=peslog(n)

Creamos una matriz de peslog(n) columnas y n filas. O(n*peslog(n))
Le asignamos 0 a cada posicion. O(n*peslog(n))
Le asignamos grupo 1 y 2 alternadamente a los guerreros, en casa pelea, que esta representada por una fila en nuestra matriz. O(n*peslog(n))
Por lo tanto, el algoritmo es O(3*n*peslog(n)). Como 3 es una constante, la ocmplejidad final es O(n*peslog(n)).
\end{verbatim}

\subsection{Extracto importante de código}
\begin{lstlisting}
while(i < filas){
	cambio = pow(2,i);
	x = 0;
	j = 0;
	while(j < n){
		if(x < cambio){
			m[i][j] = equipo;
		}else{
			equipo = ((equipo +1) % 2);
			m[i][j] = equipo;
			x = 0;
		}
		j++;
		x++;
	}
	i++;
}
\end{lstlisting}

\subsection{Experimentacion}
\begin{verbatim}
%ya esta hecha
\end{verbatim}

\end{document}

%lo dejo aca un poco

%\paragraph{ Idea: cada vez que se resuelve una pelea entre grupos, si la misma se da entre dos mitades, ya se consigue que la mitad de los luchadores hayan peleado entre si, quedando pendiente dentro del mismo grupo.}

%\begin{verbatim}
%p <- 1

%kaioKen (N: int){
%	kaioKenGeneralizada(N,1)
%}

%kaioKenGeneralizada (N: int, P: int) {
%	for (N)
%		B (P, N)
%	endfor
%	p += 1
%	if (N != 1 v 2)		
%		kaioKenGeneralizada(ceil(n/2), p)
%		kaioKenGeneralizada(floor(n/2), p)
%	endif	
%}
%\end{verbatim}


