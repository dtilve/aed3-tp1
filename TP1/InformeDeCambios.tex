
\documentclass[10pt,a4paper]{article}
\usepackage[utf8]{inputenc} % para poder usar tildes en archivos UTF-8
\usepackage[spanish]{babel} % para que comandos como \today den el resultado en castellano
\usepackage{amsmath}
\usepackage{listings}
\usepackage{color}
\usepackage{algorithm}
\usepackage{algpseudocode}
\usepackage{mathtools}
\DeclarePairedDelimiter\ceil{\lceil}{\rceil}
\DeclarePairedDelimiter\floor{\lfloor}{\rfloor}

\usepackage[conEntregas]{caratula}

\usepackage[margin=2cm]{geometry}
\usepackage{graphicx}
\usepackage{wrapfig}

\begin{document}
\par{Informe de cambios}
\\
\par{Ejercicio1:
\begin{itemize}
\item[•] Agregada sección "Demostración de optimalidad".
\item[•] Nueva toma de tiempos.
\item[•] Nueva gráfico y tabla de entrada * tiempo promedio acorde a las nuevas mediciones.
\end{itemize}
}

\par{Ejercicio2
\begin{itemize}
\item[•] Introducción a los puntos después de la definición de área de impacto.
\item[•] Demostración matemática de la complejidad lineal de genkidama.
\item[•] Nueva toma de tiempos.
\item[•] Nueva gráfico y tabla de entrada * tiempo promedio acorde a las nuevas mediciones.
\end{itemize}
}

\par{Ejercicio3
\begin{itemize}
\item[•] Cambio en la explicación de la complejidad.
\item[•] Nueva toma de tiempos en mejor casp.
\item[•] Nueva gráfico acorde a las nuevas mediciones.
\end{itemize}
}
\end{document}